\documentclass[10pt,oneside]{book}
\usepackage[utf8]{inputenc}
\usepackage{import}
\usepackage{lscape}
\usepackage{lettrine}
\usepackage[left=1.5in, right=1in, top=1in, bottom=1in]{geometry}

\usepackage{fancyhdr}
\fancyhf{}
\lhead{\leftmark}
\cfoot{\thepage}

\usepackage{graphicx}
\graphicspath{{figures/},{figures/altro/}}

% Attitude Paper Commands
\newcommand{\Q}{\mathbb{S}^3}
\newcommand{\skewmat}[1]{[#1]^\times}
\newcommand{\rmap}{\varphi}
\newcommand{\invrmap}{\varphi^{-1}}
\newcommand{\rot}{ \mathcal{R} }
\newcommand{\dq}{\delta q}
\newcommand{\q}{\textbf{q}}
\newcommand{\eq}{_\text{eq}}
\newcommand{\traj}[2][N]{#2_{0:{#1}}}
\newcommand{\pass}{{\color{green} \checkmark}}
\newcommand{\fail}{{\color{red} \ding{55}}}
\newcommand{\inframe}[2]{{}^{#1}\!#2}
\newcommand{\toframe}[3]{\inframe{#1}{#3}^{#2}}

% Distributed paper commands
\newcommand*\rotbf[1]{\rotatebox{90}{\textbf{#1}}}
\newcommand{\specialcell}[2][c]{\begin{tabular}[#1]{@{}l@{}}#2\end{tabular}}
\newcommand{\specialcellbold}[2][c]{%
	\bfseries
	\begin{tabular}[#1]{@{}l@{}}#2\end{tabular}%
}

% Bibliography
\usepackage[style=ieee]{biblatex}
\addbibresource{Dissertation.bib}
% \addbibresource{Publications.bib}

\usepackage{brian}
\usepackage{subfiles}

\usepackage[explicit]{titlesec}
\usepackage{color}
\definecolor{RIRed}{RGB}{148,19,30}
\newlength\chapnumb
\setlength\chapnumb{4cm}
\titleformat{\chapter}[block]
    {\normalfont\sffamily}{}{0pt}
    {
        \parbox[b]{\chapnumb}
        {\fontsize{120}{110}\color{RIRed}\selectfont\thechapter}%
        \parbox[b]{\dimexpr\textwidth-\chapnumb\relax}{%
        \raggedleft%
        \hfill{\huge#1}\\
        {\color{black}{\rule{\dimexpr\textwidth-\chapnumb\relax}{0.8pt}}}}
    }
\titleformat{name=\chapter,numberless}[block]
    {\normalfont\sffamily}{}{0pt}
    {
        % \parbox[b]{\chapnumb}{\mbox{}}%
        \parbox[b]{\dimexpr\textwidth-\chapnumb\relax}{%
        \huge#1\\
        % \raggedleft%
        % \hfill{\huge#1}\\
        \rule{\textwidth}{0.4pt}}
    }


\begin{document}

    \begin{titlepage}
    \begin{center}
        \vspace*{1.0in}
        % {\LARGE \textit{Thesis Proposal}\\}
        % \vspace{0.2in}
        {\Huge \textbf{
        Accelerating 
        Numerical Methods for
        Optimal Control 
        }} \\
        \vspace{0.2in}
        {\Large Brian Jackson} \\
        \vspace{0.2in}
        {\large May 2021}
        
        \vfill
        \includegraphics[width=1.0in]{RI_large.jpg}

        The Robotics Institute \\
        Carnegie Mellon University \\
        Pittsburgh, Pennsylvania 15213 \\

        \vspace{0.2in}

        \textbf{Thesis Committee}\\
        Zachary Manchester, Chair, CMU RI \\
        Michael Kaess, CMU RI \\
        Lorenz Biegler, CMU ChemEng \\
        Scott Kuindersma, Boston Dynamics \\

        \vspace*{0.5in}

        \textit{Submitted in fulfillment of the requirements for the degree of Doctor of Philosophy in Robotics }\\

    \end{center}
\end{titlepage}

    \frontmatter
    
    \newpage
    \pagestyle{empty}
    \begin{center}
    \copyright{} 2021 Brian Edward Jackson \\
    ALL RIGHTS RESERVED
    \end{center}
    
    \chapter*{Abstract}
    Many modern control methods, such as model-predictive control, rely heavily
    on solving optimization problems in real time. In particular, the ability to
    efficiently solve optimal control problems has enabled many of the recent
    breakthroughs in achieving highly dynamic behaviors for complex robotic
    systems. The high computational requirements of these algorithms demand novel
    algorithms tailor-suited to meeting the tight requirements on runtime
    performance, memory usage, reliability, and flexibility. This thesis
    introduces a state-of-the-art algorithm for trajectory optimization that
    leverages the problem structure while being applicable across a wide variety
    of problem requirements, including those involving conic constraints and
    non-Euclidean state vectors such as 3D rotations. Additionally, algorithms for 
    exposing parallelization in both the temporal and spatial domains are proposed.
    A method for efficiently improving the tracking performance of MPC controllers with 
    data from the actual system is also proposed.

    % TODO: Finish abstract

    % TODO: Dedication
    % TODO: Dedication
    \pagebreak
    \hspace{0pt}
    \vfill
    \begin{center}
    To my Dad.
    \end{center}
    \vfill
    \hspace{0pt}
    \pagebreak

    \chapter*{Biographical Sketch}
    % TODO: Biographical Sketch
    Brian Jackson was born in Provo, UT while his father finished his last year of graduate
    school at Brigham Young University, but spent most of his childhood moving around the 
    United States as his father pursued various opportunities in his career as a mechanical
    engineer. From the age of about 10, Brian knew he wanted to follow in his father's 
    footsteps and pursue a career in engineering. During his undergraduate studies in 
    the Mechanical Engineering Department at Brigham Young University, Brian's interests 
    began to focus on the interplay between computational science and engineering. He 
    discovered a passion for computational methods while working
    under the tutelage of Dr. David Fullwood, researching methods to study the 
    microstructure of materials through the analysis of electron-backscatter diffraction 
    patterns. Brian's nascent interest in robotics solidified as he worked on the BYU Mars 
    Rover capstone project during his Senior year.

    In 2018, Brian was awarded a Graduate Research Fellowship from the National Science 
    Foundation after finishing his first year of graduate school at Stanford University.
    At Stanford, Brian's broad interest in robotics began to focus on optimization-based 
    control after working with Dr. Zachary Manchester, a new professors in the Department of
    Aeronautics and Astronautics. After successfully passing Stanford's PhD 
    Qualifying Exams in 2019, Brian followed Dr. Manchester to the Robotics Institute at 
    Carnegie Mellon University during the COVID-19 pandemic in 2020. After graduation, Brian
    will be working for the space startup Albedo, applying the optimization-based control 
    methodologies to satellites collecting high-resolution satellite imagery.  

    \chapter*{Acknowledgements}
    Perhaps no-one has been as impactful to my career and education as my advisor and 
    mentor, Zac. His passion for research, aptitude for teaching, and patient mentorship 
    have inspired me in countless ways. Throughout all of the challenges of the last five 
    years, both personal and professional, Zac has provided the motivation and vision I 
    needed to continue pursing my research. 

    My wife Alyssa has supported me through it all: the stressful long-night pushes before
    deadlines, the frustration of months of research not panning out, the feelings of 
    self-doubt and inadequacy, the decision to move across the country in the middle of a 
    pandemic, and the trauma of losing a parent. I wouldn't be anywhere close to where I am
    without the years of patient counsel and perspective from my parents, and most 
    especially my mother, whose weekly phone calls have supported me far more than she 
    knows. 

    I also owe a huge thanks to the members of the Robotic Exploration Lab, most especially
    Taylor Howell, Simon Le Cleac'h, Kevin Tracy, and Jeong Hun Lee, who have provided years
    of friendship and inspiration. 

    \tableofcontents
    \listoffigures
    \listoftables

    \chapter*{Nomenclature}
    \begin{table}[h]
    \centering
    \begin{tabular}{l l}
            Acronym & Definition \\
            \midrule
            QP   & Quadratic Program \\
            SOCP & Second-Order Cone Program \\
            DDP  & Differential Dynamic Programming \\
            LQR  & Linear Quadratic Regulator \\
            TVLQR & Time-Varying LQR \\
            NLP  & Nonlinear Program \\
            SQP  & Sequential Quadratic Programming \\
            IPM  & Interior Point Methods \\
            ALM  & Augmented Lagrangian Method \\
            ADMM & Alternating Direction Method of Multipliers \\
            MPC  & Model-Predictive Control \\
            PCG  & Preconditioned Conjugate Gradient \\
            SLAM & Simultaneous Localization and Mapping \\
    \end{tabular}
    \caption{Table of Abbreviations}
    \label{tab:abbreviations}
    \end{table}
    

    \mainmatter
    \pagestyle{fancy}
    \subfile{chapters/01_intro.tex}

    \subfile{chapters/02_background.tex}

    \part{Theory and Algorithms} \label{part:theory}
    \subfile{chapters/03_altro.tex}
    \subfile{chapters/03_results.tex}

    \subfile{chapters/04_mpc.tex}
    \subfile{chapters/04_results.tex}

    \subfile{chapters/05_attitude.tex}
    \subfile{chapters/07_rslqr.tex}
    \subfile{chapters/09_mctrajopt.tex}

    \part{Application} \label{part:application}
    \subfile{chapters/06_distributed.tex}j
    \subfile{chapters/08_koopman.tex}

    \appendix
    \subfile{chapters/a1_quaternion_maps.tex}

    \printbibliography
    
\end{document}