
\documentclass[../root.tex]{subfiles}

\begin{document}
\chapter{ALTRO-C: A Fast Solver for Conic Model-Predictive Control} \label{chap:altro_c}
\chaptermark{ALTRO-C}
Building off the results of the previous chapter, here we detail the extension of the
ALTRO algorithm to second-order cone programs (SOCPs), as well as details regarding
an improved implementation that enables the state-of-the-art performance on convex 
MPC problems demonstrated in this chapter. 
This work originally appeared in \cite{jackson_ALTROC_2021}.

\section{Introduction}
    \lettrine{M}{odel-predictive control} 
    (MPC) has become a widely used approach for
    controlling complex robotic systems with several notable successes in
    recent years
    \cite{blackmore_Autonomous_2016,kuindersma_Optimizationbased_2016,carlo_Dynamic_2018}.
    By transforming the control problem into an optimization problem with an
    explicit objective and constraints, MPC can achieve desired behaviors
    while accounting for complex dynamics, torque limits, and obstacle
    avoidance. However, because these optimization problems must typically be
    solved at rates of tens to hundreds of Hertz on board the robot,
    efficient and reliable solver algorithms are crucial to their success.
    
    In practice, most MPC problems are formulated as convex optimization
    problems, since convex solvers are available that can guarantee
    convergence to a globally optimal solution, or provide a certificate of
    infeasibility if a solution does not exist. A variety of numerical
    techniques for solving such problems have been developed over the past
    several decades, and many high-quality solver implementations---both
    open-source and proprietary---are available. In the control community, a
    particular emphasis has been placed on high-performance solvers for
    quadratic programs (QPs) that are suited to real-time use on embedded
    computing hardware, including active-set \cite{ferreau_QpOASES_2014}
    \cite{kuindersma_Efficiently_2014}, interior-point
    \cite{frison_Efficient_2014,frison_Highperformance_2014, frison_HPIPM_2020}, 
    and alternating
    direction method of multipliers (ADMM) \cite{stellato_OSQP_2020} methods. There has
    also been at least one interior-point solver for second-order cone
    programs (SOCPs) developed for embedded applications \cite{domahidi_ECOS_2013}.
   

    To achieve high performance on MPC problems, a solver must a) exploit
    problem structure with sparse matrix factorization techniques, b) take
    advantage of previous solutions to ``warm start'' the current solve, and
    c) efficiently handle convex conic constraints on states and inputs. We
    present ALTRO-C, a more capable version of the ALTRO solver
    \cite{howell_ALTRO_2019} originally developed for offline solution of nonlinear
    trajectory optimization problems, that achieves all three of these
    properties. As a result, it delivers state-of-the-art performance on both
    QPs and SOCPs in MPC applications, and can easily support optimization
    over other cones in the future. To the best of our knowledge, ALTRO-C is
    the first solver specifically designed for MPC applications that can
    handle second-order cone constraints.

    In summary, our contributions include:
    \begin{itemize}
        \item A novel method for incorporating conic---including second-order
        cone---constraints into a Differential Dynamic Programming
        (DDP)-based trajectory optimization solver.
        \item An open-source solver implementation in Julia that delivers
        state-of-the-art performance for convex MPC problems with a
        convenient interface for defining trajectory optimization problems.
        \item A suite of benchmark MPC problems that demonstrate the solver's
        performance, including a satellite with flexible appendages, a
        quadruped with both linearized and second-order friction cones,
        rocket soft-landing, and manipulation with contact.
    \end{itemize}

    \section{Conic Augmented Lagrangian} \label{sec:conic-al}

    There has been great interest in recent years within the optimization
    community in optimization over cones, sometimes referred to as
    \textit{generalized inequalities}. The second-order cone,
    $\mathcal{K}_\text{soc} = \{(v,s) \in \R^{n+1} \,|\, \norm{v}_2 \leq
    s\}$, 
    (also known as the \textit{quadratic cone} or the \textit{Lorentz
    cone}) has proven particularly useful in control applications including
    the rocket soft-landing problem \cite{acikmese_Convex_2007}
    \cite{acikmese_Lossless_2013} and friction cone constraints that appear
    in manipulation or locomotion tasks \cite{lobo_Applications_1998}. Before methods
    existed for directly attacking SOCPs, many practitioners linearized this
    cone, resulting in increased problem sizes and less accurate or
    sub-optimal solutions.
    
    Most existing specialized ``conic'' solvers are based on ADMM
    \cite{garstka_COSMO_2019,odonoghue_Conic_2016} or interior-point
    \cite{domahidi_ECOS_2013} methods. Both of these approaches have tradeoffs:
    ADMM methods converge slowly (linearly) but are very warm-startable,
    while interior-point methods converge quickly (quadratically) but aren't
    well-suited to warm-starting. The augmented Lagrangian method (ALM) is an
    attractive middle-ground, offering both superlinear convergence and good
    warm-starting capabilities.
    
    Despite some theoretical work showing promising convergence analyses of
    ALM for conic programs 
    \cite{liu_Convergence_2008,shapiro_Properties_2004,sun_Rate_2008,cui_Rsuperlinear_2019,hang_Augmented_2020}
    and several ALM
    implementations for solving large-scale semi-definite programs (SDPs)
    \cite{zhao_NewtonCG_2010,li_QSDPNAL_2018}, no ALM implementation for SOCPs
    existed until very recently \cite{liang_Inexact_2020}.
    %, which demonstrated state-of-the-art performance compared to Mosek and
    %SDPT3 \cite{toh1999sdpt3} on large-scale problems.
    Building on that work, our aim is to develop an ALM SOCP solver that
    exploits the special structure of MPC problems.

    As described in Sec. \ref{sec:background_alm}, augmented Lagrangian
    methods solve constrained optimization problems by solving a series of
    unconstrained problems minimizing the
    \textit{augmented Lagrangian}:
    \begin{equation} \label{eq:augmented_lagrangian}
        \mathcal{L}_A(x) = f(x) - \lambda^T c(x) + \rho \half c(x)^T c(x),
    \end{equation}
    where $f(x) : \R^{n} \mapsto \R^{}$ is the objective function, $c(x) : \R^{n}
    \mapsto \R^{m}$ is an equality constraint function, $\lambda \in \R^{m}$ is a
    Lagrange multiplier, and $\rho \in \R$ is a penalty weight. This standard
    form can also be adapted to handle inequality constraints, as described
    in \cite{howell_ALTRO_2019} \cite{toussaint_Novel_2014}.
    
    After each unconstrained minimization of \eqref{eq:augmented_lagrangian}
    with respect to $x$, the penalty $\rho$ is increased and the Lagrange
    multiplier $\lambda$ is updated according to,
    \begin{equation} \label{eq:al_dual_update2}
        \lambda \gets \left( \lambda - \rho c(x) \right) ,
    \end{equation}
    which is equivalent to a gradient ascent step on the dual problem \cite{bertsekas_Constrained_1996}.
    Augmented Lagrangian methods are theoretically capable of superlinear
    convergence rates, but often exhibit poor ``tail-convergence'' behavior
    in practice due to ill-conditioning as $\rho$ is increased.
    %In contrast, interior-point and active-set methods can achieve local quadratic convergence rates near an optimum.
    
    Based on \cite{liu_Convergence_2008}, we generalize the augmented
    Lagrangian method to enforce general conic constraints. Equation
    \eqref{eq:augmented_lagrangian} can be rewritten as,
    \begin{equation}
        \mathcal{L}_A(x) = f(x) + \frac{1}{2 \rho} (\norm{\lambda - \rho c(x)}^2 - \norm{\lambda}^2).
    \end{equation}
    Comparing the first quadratic penalty term $\lambda - \rho c(x)$ with the
    standard dual ascent step \eqref{eq:al_dual_update2}, we see that this
    reformulation is effectively penalizing the difference between the
    current and updated Lagrange multiplier estimates.
    
    If our constraint is instead required to lie within the cone
    $\mathcal{K}$, we can modify the augmented Lagrangian penalty to penalize
    the difference between the multipliers after the updated multiplier is
    projected back into the cone,
    \begin{equation} \label{eq:conic_augmented_lagrangian}
        \mathcal{L}_A(x) = f(x) + \frac{1}{2 \rho} (\norm{\Pi_\mathcal{K}(\lambda - \rho c(x))}^2 - \norm{\lambda}^2),
    \end{equation}
    where $\Pi_\mathcal{K}(x) : \R^{p} \mapsto \R^{p}$ is the projection operator
    for the cone $\mathcal{K}$. We refer to
    \eqref{eq:conic_augmented_lagrangian} as the \textit{conic augmented
    Lagrangian}.
    
    For simple inequality constraints of the form $c(x) \leq 0$, the
    projection is onto the non-positive orthant: $\Pi_{\mathcal{K}_-}(x) =
    \min(0,x)$. Simple closed-form expressions for the projection operator
    exist for several other cones, including the second-order cone:
    \begin{equation} \label{eq:soc_projection}
        \Pi_{\mathcal{K}_\text{soc}}(x) = \begin{cases}
            0 & \norm{v}_2 \leq -s \\
            x & \norm{v}_2 \leq s  \\
            \half (1 + s/\norm{v}_2) [v^T \; \norm{v}_2]^T & \norm{v}_2 > \abs{s}
        \end{cases}
    \end{equation}
    where $v = \begin{bmatrix} x_0 \dots x_{p-1} \end{bmatrix}^T$, $s = x_p$.
    Analogous to \eqref{eq:al_dual_update2}, the dual update in the conic case
    becomes,
    \begin{equation}
        \lambda \gets \Pi_\mathcal{K}(\lambda - \rho c(x)) .
    \end{equation}
    
    This method for handling conic constraints is the key algorithmic
    development in ALTRO-C, enabling the competitive timing and convergence
    results demonstrated in Section \ref{sec:mpc_examples}.
    
    % %and the penalty term is updated as normal.
    
    
    \section{ALTRO-C Solver} 
    %     %The ALTRO solver is a fast solver for constrained, non-convex trajectory optimization that uses iterative LQR (iLQR) within an augmented Lagrangian framework to handle general equality and inequality constraints. To handle the slow tail convergence typically exhibited by augmented Lagrangian methods, ALTRO uses an active-set projected Newton method for solution polishing. The algorithm is summarized in the following paragraphs.
        
        The original ALTRO algorithm, described in Sec. \ref{chap:altro} and
        \cite{howell_ALTRO_2019}, was modified using the formulation in
        \ref{sec:conic-al} to handle second-order cone constraints and was
        renamed ALTRO-C. Analytic first and second-order derivatives of
        \eqref{eq:soc_projection} were implemented to calculate the
        expansions of \eqref{eq:conic_augmented_lagrangian} required by the
        iLQR algorithm. No changes were made to the active-set
        solution-polishing method. To our knowledge, ALTRO-C is the first
        DDP-based algorithm to support conic constraints, and one of the
        first optimal control solvers in general to support generalized
        inequalities.

        In addition to adding second-order cone constraints, the Julia
        implementation of ALTRO-C has been improved substantially from the
        original version presented in \cite{howell_ALTRO_2019}, enabling the competitive
        timing results demonstrated in Section \ref{sec:mpc_examples}. Memory
        allocations have been eliminated wherever possible, and ALTRO-C has been
        particularly optimized for small-to-medium-size problems by leveraging
        loop-unrolling and analytical linear algebra\footnote{These algorithms
        are part of the
        \href{https://github.com/JuliaArrays/StaticArrays.jl}{StaticArrays.jl}
        package}.
        If the problems are small enough, the matrix operations in the
        backward pass are stored on the stack and all operations are unrolled
        at compile time, resulting in speed improvements of 10-100x over the
        built-in heap-allocated Julia arrays that use BLAS or LAPACK linear
        algebra subroutines. On the benchmark problems included in
        \cite{howell_ALTRO_2019}, these performance enhancements achieved
        anywhere from a 65-140x improvement in runtime performance over the
        original implementation.
        %made available through packages in the Julia ecosystem.
        In addition to excellent performance, ALTRO-C provides a convenient API
        that dramatically simplifies MPC problem definition and provides
        convenient and efficient methods for updating the MPC problem between
        iterations. 
        By providing several methods for modifying the problem and
        constraints in-place, ALTRO is uniquely well-suited to solving MPC
        problems, especially since the solver is well-suited to warming-starting.
        While not as lightweight as embedded solvers such as ECOS, Julia can be
        run on ARM processors, and future work will investigate runtime performance
        on microcomputers.

        % In contrast to many modern ``direct''
        % methods for trajectory optimization, ALTRO relies on iterative LQR
        % (iLQR) to remain dynamically feasible at every iteration and exploit
        % the Markovian structure of MPC problems. At each iteration of the
        % iLQR algorithm (summarized in Algorithm \ref{alg:iLQR}), a
        % second-order Taylor series approximation of the problem is computed
        % and a backward Riccati recursion is used to compute an update to the
        % nominal (feed-forward) control trajectory and a time-varying LQR
        % (TVLQR) feedback controller. The closed-loop dynamics are then
        % simulated forward to compute an updated state trajectory.
        
        % While fast and efficient, the standard iLQR algorithm has no ability
        % to deal with constraints on the states or controls. To handle
        % constraints, ALTRO uses iLQR as the inner unconstrained solver in an
        % augmented Lagrangian method. To overcome the poor tail convergence of
        % the ALM in situations where tight solution tolerances are required,
        % ALTRO performs a final active-set projected Newton
        % ``solution-polishing'' step using a Cholesky factorization coupled
        % with iterative refinement \cite{howell_ALTRO_2019}. The key
        % algorithmic difference between the original ALTRO algorithm and
        % ALTRO-C is the ability to handle generalized inequalities using the
        % method described in Section \ref{sec:conic-al}. To our knowledge,
        % ALTRO-C is the first DDP-based algorithm to support conic
        % constraints.
        %ALTRO tends to converge quadratically to very good constraint satisfaction with just 1 or 2 of these projected Newton iterations.
    
    %     \begin{algorithm}
    %     \begin{algorithmic}[1]
    %     \caption{iLQR} \label{alg:iLQR}
    %         \Function{iLQR}{$\ell,f,X,U$} 
    %             \While{not converged}
    %                 \State $J(\delta X,\delta U) \leftarrow$ Quadratic expansion of $\ell$ at $X,U$
    %                 \State $A_{1:N}, B_{1:N} \leftarrow$ Linearize dynamics $f$ at $X,U$
    %                 \State $K_{1:N},d_{1:N} \leftarrow \textproc{TVLQR}(J,A_{1:N},B_{1:N})$
    %                 \State $\alpha \leftarrow 1$
    %                 \For{k=1:N-1}
    %                     \State $\bar{u}_k = u_k + K_k(\bar{x}_k - x_k) + \alpha d_k$
    %                     \State $\bar{x}_{k+1} \leftarrow$ $f(\bar{x}_k, \bar{u}_k)$
    %                 \EndFor
    %                 \If{line search conditions satisfied}
    %                     \State $X \leftarrow \bar{X}, U \leftarrow \bar{U}$
    %                 \Else
    %                     \State Reduce $\alpha$ and go to line 7
    %                 \EndIf
    %             \EndWhile\\
    %             \Return $X,U$
    %     \EndFunction
    %     \end{algorithmic}
    %     \end{algorithm} 
        
    
    % \subsection{Implementation Details} \label{sec:implementation}
    % The ALTRO solver was adapted to support second-order cone constraints
    % using the augmented Lagrangian formulation introduced in Section
    % \ref{sec:conic-al}. 
    

\end{document}