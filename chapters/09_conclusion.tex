\documentclass[../root.tex]{subfiles}

\begin{document}
\chapter{Conclusion} \label{chap:conclusion}
This chapter provides a timeline for the work proposed in Chapters \ref{chap:blimp}
and \ref{chap:parallel}, as well as a summary of the content presented in this 
thesis. 

\section{Timeline}
Below is the proposed timeline for the work outlined in Chapters \ref{chap:blimp}-
\ref{chap:parallel}.
\begin{table*}[h]
    \begin{tabular}{l p{0.25\linewidth} p{0.25\linewidth} p{0.25\linewidth}}
        Semester & Academic & Research & Deliverables \\
        \midrule
        Summer 2021 
            & 
            & Internship at Optimus Ride. 
            & C++ Version of ALTRO, w/ possible IROS 2021 paper \\
        Fall 2021   
            & Waive 16-811. Take Zac's dynamics class. 
            & Nested dissection / parallel trajectory optimization.
            & Paper submission to RSS / ICRA 2022. \\
        Spring 2022 
            & 
            & Internship at NASA JPL.
            & Paper on dynamics emulation. \\
        Summer 2022
            & 
            & GPU implementation of trajectory optimization algorithm.
            & Defense and paper submission to IROS 2022. \\
        Fall 2022
            & TA 16-745 for the second time.
            & Write thesis. 
            & Defense and finished thesis.
    \end{tabular}
\end{table*}

\section{Summary}
Robots have considerable potential to fullfill many meaningful roles 
in our society, across a broad range of applications and sectors of 
our economy. As robots move into the applications of today and tomorrow,
they need to be endowed with a greater ability to make meaningful 
decisions and operate at the limits of their desired objectives, 
whether that be gathering scientific information on another planet,
unloading pallets, delivering mail, or providing helpful care to 
the sick or the elderly. A key component in all of these applications 
is the ability to make and follow efficient motion plans.

The work detailed in this thesis has made significant contributions to
progressing the promising field of trajectory optimization. It has pushed the
limits of computational performance, allowing robots to plan faster or take
more into consideration in the plans they generate. It has provided
meaningful theoretical contributions that will aid the next generation of
roboticists develop better tools for systems that operate freely in
three-dimensional space. Building off these successes, we have proposed two
new branches of research moving forward, with the goals of again pushing the
limits of computational performance for these critical algorithms, and
demonstrating their usefulness on real hardware. We believe that, when
accomplished, these contributions to the field of robotics will have
furthered the field of optimal control and trajectory optimization in
meaningful ways, and thereby meet the high expectations of the work of a
doctoral student in Robotics at Carnegie Mellon University.


\end{document}