\documentclass[../root.tex]{subfiles}

\begin{document}
\chapter{Parallelizing Trajectory Optimization} \label{chap:parallel}

\lettrine{M}{odern} implementations of trajectory optimization algorithms
such as ALTRO (see Chapters \ref{chap:altro}-\ref{chap:attitude},
\cite{howell_ALTRO_2019}), OCS2 \cite{farshidian_OCS2_}, Frost
\cite{hereid_FROST_2017}, Drake \cite{tedrake_Drake_}, or Crocoddyl
\cite{mastalli_Crocoddyl_2019} have acheived impressive performance results.
However, as we continue to put these algorithms out into the real world and
apply them on increasingly difficult problems, the demand for even faster
algorithms will continue to rise. Faster trajectory optimization algorithms
can be helpful in at least two ways: 1) Faster replanning enables more
adaptive / reactive behavior, or alternatively allows for the online solution
of more complicated problems. In order to remain computationally tractable,
many practitioners make strong assumptions on the low-level MPC controllers
that are solved online. Relaxing these assumptions could open up new
frontiers in terms of both performance and safety of these systems. 2) Faster
offline planning allows for a larger variety of solutions to be found and
enables quicker turn-around time for practitioners designing the
trajectories. As most practitioners know, nonlinear trajectory optimization
is an art form where the available modes of expression include initial
guesses, cost shaping, and constraint modification / problem reformulation.
Getting the solve time out of the way of the designer allows for more rapid
feedback and better convergence towards high-quality trajectories.
Alternatively, it allows for more runs to be executed in a Monte-Carlo-style
approach, as is often used in space trajectory design where the optimized
trajectories may appear chaotic and/or extremely sensitive to the initial
guess
\cite{grebow_MCOLL_2017,englander_Multiobjective_2015,tracy_LowThrust_2021}.
Reducing the runtime by an order of magnitude or more means that solutions
that used to take weeks or months take days or even hours, saving massive
amounts of compute time.

While some of these solver implementations have leveraged parallelization in
parts of the classical algorithms such as direct collocation or DDP, the
algorithms themselves either remain inherently serial---in the case of DDP---or 
rely on single-threaded NLP solvers---in the case of direct collocation. As
we're slowly approaching the limit of what single-threaded applications can
achieve, it's apparent that further dramatic speed improvements can only be
achieved through parallelization. As a result, there has naturally been an
increased interest within the community to leverage parallelization. Much of
this work has only demonstrated parallelization in theory, such as the use of
ALM to parallelize an SQP algorithm \cite{kouzoupis_block_2016},
parallelization of the quadratic MPC problem \cite{frasch_parallel_2015}, or
parallelization of function and gradient evaluations using a discrete
algebraic equation (DAE) solver \cite{leineweber_efficient_2003}. The most
common approach is to simply parallelize the dynamics, cost, and derivative
information required during the algorithms \cite{giftthaler_Family_2017,
mastalli_Crocoddyl_2019, antony_Rapid_2017, neunert_WholeBody_2018}. In an
attempt to open up further parallelism within shooting-based methods like
DDP, many researchers have investigated multiple-shooting techniques where
the trajectory is split into sub-trajectories that can be optimized in
parallel, while maintaining some continuity constraints between the
sub-trajectories \cite{bock_Multiple_1984,giftthaler_Family_2017}. Taking
this approach to the limit, \cite{plancher_Performance_2020} ported this
algorithm onto GPUs and FPGAs (field-programmable gate arrays), demonstrating
some sizeable performance gains, but found that this method displayed an
undesireable tradeoff between parallelism and convergence. More promising
results were recently found using ADMM (Alternating Direction Method of Multipliers)
on large-scale distributed MPC applications \cite{alonso_Effective_2021}.
Another impressive recent result found trajectories for systems with contact 
(e.g. walking robots) by smoothing the contact forces, using a clever formulation 
of the dynamics, using a parallel factorization method for the resulting 
block-tridiagonal system. They achieved a 30x runtime improvement over a
similar implementation on a multi-core CPU \cite{pan_GPUbased_2019}.

Rather than trying to parallelize gradient or Newton-based methods, some researchers 
have taken sampling-based approaches, which are naturally better suited to 
massive parallelization. Some impressive results using Model Predictive Path Integral 
Control (MPPI) have been demonstrated on small-scale off-road racing, where the 
car is able to ``adapt'' its strategy to learn the best actions to corner on dirt at 
high speeds \cite{williams_Aggressive_2016}. Several papers have been published on 
various methods of mapping variants of this algorithm to GPUs 
\cite{williams_Model_2015,phung_Model_2017}. The key drawbacks of these approaches 
is that they don't scale well to problems with higher dimensions and don't work 
well for systems that are highly unstable, where small variations in the control input 
result in massive changes in the trajectory. Another interesting approach uses genetic 
algorithms to solve the MPC problem, which again can be mapped extremely well to a GPU 
\cite{hyatt_Realtime_2017,hyatt_RealTime_2020}. 


\section{Preliminary Work}

In a departure from most of the current work attempting to parallelize
trajectory optimization problems, which look for ways to map existing,
well-proven Newton-based algorithms such as DDP onto accelerated hardware
like GPUs \cite{plancher_Performance_2020, giftthaler_Family_2017} our
approach is to take a step back and investigate naturally parallelizable
Newton or gradient-based algorithms that may map well onto the trajectory
optimization problem, similar to \cite{alonso_Effective_2021,
pan_GPUbased_2019}. In our research so far, we are investigating a couple
different options, detailed in the following sub-sections.

\subsection{Preconditioned Conjugate Gradient}
Preconditioned conjugate gradient (PCG) is a popular method for approximately
solving linear systems \cite{nocedal_Numerical_2006}. Instead of direct methods
which factorize a matrix into diagonal or triangular matrices that can then 
be easily solved (e.g. Cholesky, LU, LDL, QR, etc.), indirect methods such as
PCG solve an unconstrained optimization problem where the solutions 
converge to the solution of the linear system $Ax = b$. 
These methods are dominated by 
sparse matrix-vector multiplications, which easily leverage sparsity and 
map well onto parallel computing hardware. 

One of the disadvantages of these methods, including PCG, is their 
sensitivity to numerical conditioning. To mitigate this issue, PCG 
allows for a \textit{preconditioner} $P$ such that the new linear system
being solved is 
\begin{equation}
    PAx = Pb.
\end{equation}
The goal is then to find a good preconditioner $P$ that is easily invertible
and that improves the numerical conditioning of the linear system, 
especially one whose inverse can be computed in parallel. The use of preconditioned
Krylov methods was recently used for nonlinear MPC \cite{knyazev_Sparse_2016}. 

\subsection{Nested Dissection}
The trajectory optimization problem can be viewed as a graph that is almost
a simple path graph, as shown in Fig. \ref{fig:LQR_graph}. This graphical 
representation of the LQR problem isn't new, and has been studied by the 
state estimation and mapping community for several years, where they cast the 
control problem as a probabilistic inference problem \cite{toussaint_Robot_2009a}.
In recent years, the SLAM (simuntaneous localization and mapping) community has 
focused on efficient solvers for inference on the descriptive ``factor graph'' that
encodes the relationships between states and measurements across time 
\cite{dellaert_Factor_2017,kaess_iSAM2_2012}. Using state-of-the-art solvers such as
iSAM2 \cite{kaess_iSAM_2008,kaess_iSAM2_2012}, this community has shown some promising
results solving a reformulation of the LQR problem 
\cite{yang_Equality_2020,dong_Motion_2016,mukadam_Gaussian_2016}. However, most of the 
approaches still use sequential CPU-based solvers like \cite{kaess_iSAM2_2012}. 
Some recent work by this community shows some extremely promising computational results 
using Gaussian Belief Propagation (GBP), solved using novel Intelligence Processing Units 
(IPUs) \cite{davison_FutureMapping_2019,ortiz_Bundle_2020}.

Nested dissection is one of the many algorithms used by the numerical
computation community to solve problems on graphs, and one that shows promise
for solving the LQR sub-problem that arises in many algorithms for nonlinear
trajectory optimization. The algorithm, originally introduced by Alan George
in 1973 \cite{george_Nested_1973}, was originally designed for solving square
mesh grids and later generalized to generic planar graphs
\cite{lipton_Generalized_1979}. It works by recursively dividing the mesh grid
into quadrants. After solving the lowest level of the recursive partitioning
using small, dense matrix factorizations, the information is propagated back
through the recursion using Shur complements. The algorithm has been
extensively studied through the last several decades, and is summarized well
in \cite{khaira_Nested_}. Unsuprisingly, the algorithm was also picked up by
the SLAM community back in 2010 \cite{ni_Multilevel_2010}. The key idea is to
identify efficient ``separating'' variables that, when eliminated, result in
disconnected components in the underlying graph structure, equivalent to
solving block-diagonal linear system. For the LQR problem (see Fig.
\ref{fig:LQR_graph}), it is easy to see that by selecting the dual variables
associated with the dynamics constraints (i.e. $\lambda_2, \dots, \lambda_N$) the
graph separates cleanly.

Our preliminary efforts at implementing this algorithm show promise, but significant work
still remains to find an efficient way to implement the unique structure of the LQR problem
within the context of nested dissection while efficiently leveraging the resulting 
parallelism. 

\begin{figure}
    \centering
    \includegraphics{figures/parallel/lqr_graph2.tikz}
    \caption{LQR graph}
    \label{fig:LQR_graph}
\end{figure}

\section{Future Work}

Given that the preliminary results are very early-stage, there is substantial work to 
be done on this project. We plan to first implement the nested dissection algorithm for 
LQR problems on a multi-core CPU, leveraging SIMD instructions for increased parallelization.
This will be compared against standard Riccati recursion, which has linear complexity 
in the time horizon and cubic complex in the state and control dimensions. This then 
needs to be incorporated into a nonlinear optimization method like SQP or ALM, or perhaps
a combination of the two, and compared against other standard algorithms like direct 
collocation and DDP.

There is also potential for using nested dissection or similar methods to extend the work 
to multi-body systems represented with maximal coordinates, since these systems should have
a meshgrid-like underlying graph structure \cite{brudigam_LinearTime_2020}. This approach 
may be able to jointly parallelize over both the trajectory optimization problem and 
the dynamics evaluations, and lend itself nicely to an easy-to-use software library 
that works for arbitrary multi-link robots, although the ability to handle closed-loop 
kinematics is uncertain.

There also might be some interesting avenues of research investigating the use of the 
related multigrid methods used by the FEA (finite element analysis) community, which 
take a similar hierachical approach to solving rectangular meshes. Although this might 
be beyond the scope of this work, an interesting application could be including a 
small-scale CFD (computational fluid dynamics) simulation to account for complicated 
aerodynamics for aerial systems like quadrotors flying near walls, through the wake 
of other quadrotors, or aggresive aerial maneuvers for fixed-wing airplanes.



\end{document}