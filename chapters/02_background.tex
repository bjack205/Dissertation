\documentclass[../root.tex]{subfiles}

\begin{document}
\chapter{Background}

\section{Notation}
    For a function $f(x,u)$, we define $f_x \equiv \pdv*{f(x,u)}{x}|_{x_k,u_k}$, 
    $f_{xx} \equiv \pdv*[2]{f(x,u)}{x}|_{x_k,u_k}$, and
    $f_{xu} \equiv \pdv*{f(x,u)}{x}{u}|_{x_k,u_k}$. We 
    define a vertical concatenation, $(A,B,C) \equiv [A^T B^T C^T]^T$.
    The set of natural numbers from one to a positive integer $p$ is denoted 
    $\mathbb{N}_p \equiv \{1,\dots,p\}$. Let $\zeros_n \in \R^n$ be the zeros 
    vector of length $n$ and $I_n \in \R^{n \times n}$ be the $n$-by-$n$ 
    identity matrix.

\section{Trajectory Optimization}
So far we have discussed methods for solving trajectory optimization problems
without formally defining the trajectory optimization problem. While
trajectory optimization problems can take a variety of forms, in robotics we
often deal with problems of the following form:

\begin{mini}[2] 
	{x(t),u(t)}{\ell(x(t_f)) + \int_0^{t_f} \ell(x(t),u(t)) dt }{}{} 
	\addConstraint{\dot{x} = f(x,u)} 
	\addConstraint{x(0)}{=x_\text{init}}
	\addConstraint{g(x(t),u(t))}{=0} 
	\addConstraint{h(x(t),u(t))}{\leq0}
\label{opt:trajopt}, \end{mini} 
where $x(t)$ and $u(t)$ are the state and control trajectories from time $0$
to $t_f$, respectively. Unlike most optimization problems,
\eqref{opt:trajopt} is an infinite-dimensional nonlinear optimization
problem, since the optimization variables $x(t)$ and $u(t)$ are arbitrary
continuous-time trajectories. Adding further complexity, the dynamics
constraint $\dot{x} = f(x,u)$ constrains the derivatives of the optimization
variables, implying the need to solve differential equations.

Rather than attempting to solve \eqref{opt:trajopt} by deriving the
analytical necessary conditions for optimality in continuous time (i.e. an
``indirect'' method), nearly all modern practitioners first discretize it,
converting it into a finite-dimensional problem optimizing a discrete set of
samples along the continuous state and control trajectories (i.e. a
``direct'' method). 
The resulting finite-dimensional optimization problem takes the following form:
\begin{mini}[2]
	{x_{1:N},u_{1:N-1}}{\ell_N(x_N) + \sum_{k=1}^{N-1} \ell_k(x_k,u_k,\Delta t) }{}{}
	\addConstraint{x_1 = x_\text{init}}
	\addConstraint{f(x_{k+1}, u_{k+1}, x_k, u_k,\Delta t) = 0,\; k \in \mathbb{Z}^+_{N-1}}
	\addConstraint{g_k(x_k,u_k)}{=0,\; k \in \mathbb{N}_N }
	\addConstraint{h_k(x_k,u_k)}{\leq0,\; k \in \mathbb{N}_N }
	\label{opt:discrete_trajopt},
\end{mini}
where $x_k \in \R^{n}$ and $u_k \in \R^{m}$ are the state and control variables
at time step $k$, $N$ is the number of time steps (a.k.a ``knot points''),
$\Delta t_k \equiv t_{k+1} - t_k$ is the time steps, and $\mathbb{N}_k \equiv 
\{1,2,\dots,k\}$. Overloading notation, here $f$ represents the discretized
dynamics, potentially using either explicit or implicit integration methods,
while $\ell_k$ is an approximation of the integrated ``stage'' cost in
\eqref{opt:discrete_trajopt}. All of the functions $\ell_k$, $\ell_N$, $f$,
$g_k$ and $h_k$ are assumed to be nonlinear and have continuous
second-derivatives, unless specified otherwise.

Moving forward, ``trajectory optimization problem'' will refer to a finite-dimensional
nonlinear optimization problem of the form \eqref{opt:discrete_trajopt}.

\section{The Linear Quadratic Regulator}
    The Linear Quadratic Regular (LQR) problem is a canonical problem in the
    theory of optimal control, partially due to the fact that it has analytical
    solutions that can be derived using a variety of methods, and from the fact
    that LQR is an extremely useful tool in practice. We present derivations for
    both continuous-time and discrete-time LQR. The ideas and concepts from these
    derivations will appear frequently throughout the rest of this work.

    The continuous-time LQR problem is formulated as
    \begin{mini}[2]
        {u(t)}{\frac{1}{2}x^T(t_f) Q(t_f) x(t_f) + 
            \frac{1}{2}\int_0^{t_f} \left(x^T(t) Q(t) x(t) + u^T(t) R(t) u(t) \right) dt}{}{}
        \addConstraint{\dot{x}(t)}{= A(t)x(t) + B(t)u(t)}
    \end{mini}
    where $R$ is a real, symmetric, positive-definite matrix and $Q$ is a real,
    symmetric, positive semi-definite matrix.

    Using equivalent methods to those used to transform \eqref{opt:trajopt} into \eqref{opt:discrete_trajopt},
    the discrete-time LQR problem can be written:
    \begin{mini}|s|
        {u(t)}{\frac{1}{2}x^T_N Q_N x_N + 
            \frac{1}{2}\sum_{k=0}^{N-1} x_k^T Q_k x_k + u_k^T R_k u_k }{}{}
        \addConstraint{x_{k+1}}{= A_k x_k + B_k u_k}
        \label{opt:discrete_LQR}
    \end{mini} 

    \subsection{The Hamilton-Jacobian-Bellman Equation}
        The Hamilton-Jacobi-Bellman equation is an important equation in
        optimal control theory that states the necessary conditions for
        optimality for a continuous-time system. We define the cost function
        to be
        \begin{equation}
            J(x(t),u(t),t) = \ell(x(t_f),t) + \int_0^{t_f} \ell(x(t),u(t),t) dt
        \end{equation}
        and the cost-to go: 
        \begin{equation}
            J^*(x(t),t) = \min_{u(t)} J(x(t),u(t),t) .
        \end{equation}
        We now define the Hamiltonian as 
        \begin{equation}
            \mathcal{H}(x(t),u(t),J_x^*,t) = \;\ell(x(t),u(t),t)
                + J_x^{*T}(x(t),t)[f(x(t),u(t),t]
        \end{equation}
        where $J_x^* = \pdv{J^*}{x}$ and $f(x(t),u(t),t)$ are the dynamics.  

        The Hamilton-Jacobi-Bellman equation is then
        \begin{equation} \label{eq:HJB}
            0 = J_t^*(x(t),t) + \mathcal{H}(x(t),u^*(t),J_x^*,t)
        \end{equation}

        We now use these to solve the continuous LQR problem. We form the Hamiltonian
        \begin{multline} \label{eq:LQR_hamiltonian}
            \mathcal{H}(x(t),u(t),J_x^*,t) = x^T(t) Q(t) x(t) + u^T(t) R(t) u(t) 
            + J_x^{*T}(x(t),t)[A(t)x(t)+B(t)u(t)]
        \end{multline}
        and minimize it with respect to $u(t)$ by setting $\pdv*{\mathcal{H}}{u} = 0$:
        \begin{equation}
            \pdv{\mathcal{H}}{u} = Ru + B^T J_x^* = 0.
        \end{equation}
        Solving with respect to $u$ yields the optimal control 
        \begin{equation}
            u^* = -R^{-1}B^T J_x^*
        \end{equation}
        which is globally optimal since $\pdv*[2]{\mathcal{H}}{u} = R(t)$ is positive definite.

        Substituting the optimal control back into our Hamiltonian \eqref{eq:LQR_hamiltonian}:
        \begin{equation}
            \begin{aligned}
                \mathcal{H}(x,u^*,J_x^*,t) =& \frac{1}{2} x^T Q x 
                    + \frac{1}{2} J_x^{*T} B R^{-1}B^T J_x^* 
                    + J_x^{*T} [Ax - B R^{-1} B^T J_x^*] \\
                =& \frac{1}{2} x^T Q x 
                    - \frac{1}{2} J_x^{*T} B R^{-1}B^T J_x^* 
                    + J_x^{*T} Ax
            \end{aligned}
        \end{equation}

        We're now ready to use the Hamilton-Jacobi-Bellman equation \eqref{eq:HJB}.
        Since the HJB equation is a first-order partial differential equation of the
        minimum cost, we need to guess a solution, which we assume to be quadratic:
        \begin{equation}
            J^*(x(t),t) = \half x^T(t) K(t) x(t)
        \end{equation}
        where $K$ is a symmetric positive-definite matrix.

        Plugging these into the HJB equation we get:
        \begin{equation} \label{eq:LQR_presolution}
            \begin{aligned}
                0 = \half x^T \big(\dot{K} + Q - K^T B R^{-1}B^T K + 2 K A \big) x \\
            \end{aligned}
        \end{equation}
        Leveraging the symmetry of quadratic forms and the fact that
        \eqref{eq:LQR_presolution} must be equal to zero for all $x(t)$ we arrive at
        the Riccati equation:
        \begin{equation}
            0 = \dot{K} + Q - K^T B R^{-1}B^T K + K A + A^T K.
        \end{equation}
        When solved for $K(t)$ using a specialized Riccati solver, the optimal control law is given by
        \begin{equation}
            u^*(t) = -R^{-1}(t)B^T(t)K(t)x(t).
        \end{equation}

    \subsection{Discrete Riccati Recursion} \label{sec:discrete_LQR}
        We now shift our focus to solving the discrete-time LQR problem
        \eqref{opt:discrete_LQR}. We start by defining the discrete cost-to-go function
        \begin{equation}
            V_k(x) = \min_{u_i,\dots,u_{N-1}} \half x_N^T Q x_N 
                + \half \sum_{i=k}^{N-1} x_i^T Q_i x_i + u_i^T R_i u_i
        \end{equation}
        subject to the dynamics: $x_{k+1} = A_k x_k + B_k u_k$. This gives
        the cost of starting at a particular state and moving towards the
        goal in an optimal manner. The cost of the entire optimal trajectory
        is therefore $V_1(x_1)$. From the Bellman equation and the principle
        of optimality we can re-define the value function in a more
        convenient, recursive form:
        \begin{equation}
            V_k(x) = \min_{u_k} \half x_k^T Q_k x_k 
                + \half u_k^T R_k u_k 
                + V_{k+1}(A_k x_k + B_k u_k)
        \end{equation}
        which simply states that the cost-to-go is the cost incurred
        for the current decision, plus the cost-to-go of where our current
        decision takes us (by simulating our dynamics forward one time
        instance). For convenience, we define the action-value function
        $Q(x_k,u_k)$ to be the value being minimized:
        \begin{equation}
            V_k(x) = \min_{u_k} Q(x_k,u_k)
        \end{equation}
        We also assume that $V_k(x)$ is a quadratic form, i.e. 
        \begin{equation}
            V_k(x) = \half x_k^T P_k x_k
        \end{equation}
        so that the action-value function takes the following form:
        \begin{equation} \label{eq:LQR_Q}
            Q(x_k,u_k) = \half x_k^T Q_k x_k + \half u_k^T R_k u_k 
                + \half (A_k x_k + B_k u_k)^T P_{k+1} (A_k x_k + B_k u_k)
        \end{equation}

        Since there are no controls to optimize at the last time step, we note that
        \begin{equation} \label{eq:LQR_termcost}
            V_N(x) = \half x_N^T Q_N x_N
        \end{equation}
        Now that we know the terminal cost-to-go, we can find the optimal
        cost-to-go for all time steps once we have a recurrence relation that
        gives $V_k$ as a function of $V_{k+1}$. We start by optimizing the
        action-value function at time step $k$, which has the following
        first-order necessary condition:
        \begin{equation}
        \begin{aligned}
            \pdv{Q}{u} =& 0 \\
            =& R_k u_k + B^T P_{k+1}(A_k x_k + B_k u_k) 
        \end{aligned}
        \end{equation}
        Solving for $u$ we find the optimal control trajectory:
        \begin{equation} \label{eq:LQR_gain}
            \begin{aligned}
            u_k^* =& -(R_k + B_k^T P_{k+1} B_k)^{-1} B_k^T P_{k+1} A_k x_k \\
            &= -Q_{uu}^{-1}Q_{ux} x_k \\
            &= -K_k x_k
            \end{aligned}
        \end{equation}
        where $Q_{uu} = \pdv*[2]{Q}{u}$ and $Q_{ux} = \pdv*{Q}{u}{x}$. 
        % We introduce this notation for easy comparison to the more
        % complicated DDP derivation in the following sections.

        With an optimal feedback policy, we now substitute
        \eqref{eq:LQR_gain} into \eqref{eq:LQR_Q} to get the cost-to-go:
        \begin{equation}
            \begin{aligned}
                V_x(x) =& Q(x_k,u_k^*) \\
                =& \half \big(x_k^T Q_k x_k + x_k^T K_k^T R_k K_k x_k  
                +  x_k^T(A_k - B_k K_k)^T P_{k+1} (A_k - B_k K_k )x_k \big) \\
                =& \half x^T \big( Q_k +  A_k^T P_{k+1} A_k  
                + K_k^T (R_k + B^T P_{k+1} B_k) K_k  \\
                &- K_k^T B_k^T P_{k+1} A_k - A_k P_{k+1} B_k K_k \big)x_k \\
                =& \half x_k^T P_k x_k
            \end{aligned}
        \end{equation}
        where, after substituting in \eqref{eq:LQR_gain} and simplifying,
        \begin{equation}
            P_k = Q_k +  A_k^T P_{k+1} A_k  
                    - A_k^T P_{k+1}^T B_k (R_k + B_k^T P_{k-1} B_k)^{-1} B_k^T P_{k+1} A_k.
        \end{equation}
        This establishes the recurrence relation between $V_k$ and $V_{k+1}$,
        which can be used to recursively calculate the cost-to-go for the
        entire trajectory, along with the optimal feedback control gains
        $K_k$.


\section{Nonlinear Programming}
    As mentioned in Sec. \ref{sec:lit_review}, direct methods solve the
    discretized trajectory optimization problem \ref{opt:discrete_trajopt}
    using generic nonlinear programming solvers. A wide variety of methods
    exist for solving this difficult class of optimization problems. The
    following sections provide some brief background on a sampling of the
    most common methods. In this section, we refer to the following generic
    NLP:

    \begin{mini}[2] 
        {x}{f(x)}{}{}
        \addConstraint{g(x) = 0}
        \addConstraint{h(x) \leq 0}
        \label{opt:NLP}
    \end{mini}
    where $f : \R^n \mapsto \R$, $g : \R^n \mapsto \R^m$, and $h : \R^n \mapsto
    \R^p$ are all assumed to be twice differential nonlinear functions. The first-order 
    optimal conditions for this problem (i.e. the KKT conditions) are:
    \begin{subequations}
        \begin{align}
            \nabla_x f(x) + \nabla_x g(x)^T \lambda - \nabla_x h(x)^T \mu &= 0 && \text{stationarity}\\
            g(x) &= 0 && \text{primal feasibility}\\
            h(x) &\leq 0 && \text{primal feasibility}\\
            \mu_i &\geq 0, \; i \in \mathbb{N}_p && \text{dual feasibility}\\
            \mu_i h(x)_i &=0, \; i \in \mathbb{N}_p && \text{complimentarity}.
        \end{align}
        \label{eq:nlp_kkt}
    \end{subequations}

\subsection{Penalty Methods} One of the simplest methods for handling
    equality-constrained NLPs (e.g. $p$ = 0), penalty methods solve an
    unconstrained optimization of the following form: \begin{mini}[2] {x}{f(x) +
    \rho \varphi(g(x))}{}{} \end{mini} where $\varphi(x) : \R^m \mapsto \R$ is
    some penalty function, often the quadratic penalty function $\varphi(x) = x^T
    x$. These methods are extremely easy to implement, but require that the
    penalty weight $\rho$ be increased to infinity to achieve zero constraint
    violation, which leads to severe issues with numerical conditioning when
    implemented in a computer. One way around this is to use an ``exact''
    penalty method using the $L_1$-norm, which is not continuously
    differentiable \cite{nocedal_Numerical_2006}. Penalty methods are therefore
    rarely used in practice, especially since augmented Lagrangian methods are
    simple extention to penalty methods that have dramatically better theoretical
    and practical properties.


\subsection{Augmented Lagrangian Method} \label{sec:background_alm}
    The Augmented Lagrangian method (ALM) is straightforward improvement upon the 
    simple penalty method. ALM minimizes the \textit{augmented Lagrangian}
    \begin{equation}
        \mathcal{L}_A(x, \lambda) = f(x) - \lambda^T g(x) + \frac{\rho}{2} g(x)^T g(x)
    \end{equation}
    with respect to the primal variables $x$, holding the dual variables $\lambda$ and 
    penalty parameter $\rho$ constant. After converging to a local optima $x^*$, 
    $\lambda$ and $\rho$ are updated using the following update rules:
    \begin{equation} \label{eq:al_dual_update}
        \lambda^+ = \lambda - \rho \, g(x^*)
    \end{equation}
    \begin{equation}
        \rho^+ = \phi \rho
    \end{equation}
    where $\phi \in \R > 1$ is some geometric factor, typically around 10. After updating 
    the dual variables and penalty parameter, the augmented Lagrangian is once again 
    minimized, using the solution from the previous iteration $x^*$ as the initial guess.

    Inequality constraints can be handled in a variety of ways, but we've found
    the version set forth in \cite{toussaint_Novel_2014} to work
    well in practice. We extend the definition of the augmented Lagrangian to be
    \begin{equation}
        \mathcal L_A(x; \lambda, \rho) = f(x) + \lambda^T g(x) + \mu^T h(x) 
            + \frac{\rho}{2} g(x)^T g(x) + \half h(x)^T I_\rho h(x)
    \end{equation}
    where 
    \begin{equation} \label{eq:penalty_matrix}
        I_\rho = \begin{cases}
            0 & \text{if } h(x)_i < 0 \; \wedge \; \mu_i > 0 \\
            \rho & \text{otherwise}
        \end{cases}
    \end{equation}

    The dual update for the inequality constraints simply projects the update multipliers 
    back into the positive orthant:
    \begin{equation}
        \mu^+ = \text{max}(\mu + \rho \, h(x), 0)
    \end{equation}
    This is done to ensure that the dual feasibility condition is always
    satisfied, while allowing the update to decrease the value of $\mu_i$ to 0
    (to satisfy the complimentarity condition) when the i-th constraint becomes
    satisfied.

\subsection{Sequential Quadratic Programming}
    Sequation quadratic programming (SQP) is a powerful method for solving nonlinear optimization 
    problems. Many variations exhist, but typically solve a quadratic program (QP) of the 
    following form:
    \begin{mini}[2]
        {x}{\half x^T P x + q^T}{}{}
        \addConstraint{A x + b = 0}
        \addConstraint{C x + d \leq 0}
    \end{mini}
    where $P \in \R^{n \times n}$ is an approximation of the Hessian of the Lagrangian,
    $q \in \R^n$ is gradient of the Lagrangian, and 
    $A \in \R^{m \times n}, b \in \R^m$,
    $C \in \R^{p \times n}, d \in \R^p$ are
    the first-order approximations of $g$ and $h$, respectively. 

    SQP methods tend to be very amenable to warm-starting, especially if a good guess of the
    active set of inequality constraints at the optimal solution is known apriori.
    Good SQP implmentations tend to require lots of careful globalization methods to ensure
    convergence, and tend to be fairly complicated to implement from scratch. The commercial
    SNOPT solver \cite{gill_SNOPT_2005} is the most commonly used SQP solver. For more details,
    see Chapter 18 of \cite{nocedal_Numerical_2006}.

\subsection{Interior Point Method}
    Interior point methods (IPM) have increasingly become the algorithm
    of choice for solving convex optimization problems, including LPs, QPs
    \cite{frison_HPIPM_2020}, and SOCPs
    \cite{domahidi_ECOS_2013}. Extensions of IPM to nonlinear programming have
    also shown impressive performance, but like SQP requires carefully
    implemented globalization strategies to ensure robust convergence to a local
    minima.

    Interior point methods solve a series of reformulated problems of the form:
    \begin{mini}
        {x,s}{f(x) - \rho \sum_{i=1}^p \log{s_i}}{}{}
        \addConstraint{g(x) = 0}
        \addConstraint{h(x) + s = 0}
    \end{mini}

    The KKT conditions of this problem are 
    \begin{subequations}
        \begin{align}
            \nabla f(x) + \nabla g(x)^T \lambda + \nabla h(x)^T \mu &= 0\\
            g(x) &= 0 \\
            h(x) + s &= 0 \\
            s, \mu &\geq 0 \\
            \mu_i - \rho \, s_i^{-1} &= 0, \; i \in \mathbb{N}_p 
        \end{align}
    \end{subequations}
    The last equation can be written the more convenient matrix form:
    \begin{equation} \label{eq:ipm_complimentarity}
        \mu - \rho I(s)^{-1} e = 0
    \end{equation}
    where $I(x)$ is a diagonal matrix with the elements of the vector $x$ on the
    diagonal, and $e \in \R^p = [1, \dots 1]^T$ is the all ones vector. Comparing
    these with \eqref{eq:nlp_kkt} we see they are almost identical, with the exception
    of the complimentarity condition, which is equal to the penalty parameter
    $\rho$ instead of zero. Thus, on each iteration of IPM, a relaxed version of
    the original KKT conditions is solved, achieving convergence to the true
    minima as $\rho \rightarrow 0$.

    Taking a first-order Taylor series approximation of these conditions, we
    obtain the following linear system:
    \begin{equation}
        \begin{bmatrix}
            \nabla^2 \mathcal{L} & 0 & \nabla g^T & \nabla h^T \\
            0 & I(\mu) & 0 & I(s) \\
            \nabla g & 0 & 0 & 0 \\
            \nabla h & I & 0 & 0 \\
        \end{bmatrix}
        \begin{bmatrix}
            \Delta x \\
            \Delta s \\
            \Delta \lambda \\
            \Delta \mu\\
        \end{bmatrix} = -
        \begin{bmatrix}
            \nabla f(x) + \nabla g(x)^T \lambda + \nabla h(x)^T \mu \\
            I(s) \mu - \rho e \\
            g(x) \\
            h(x) + s \\
        \end{bmatrix}
    \end{equation}
    where the second line is obtained by multiplying \eqref{eq:ipm_complimentarity} by $I(s)$.
    This system can be converted into an equivalent symmetric system of equations by multiplying
    the second block row by $I(s)^{-1}$ and defining $\Sigma = I(s)^{-1} I(\mu)$:
    \begin{equation}
        \begin{bmatrix}
            \nabla^2 \mathcal{L} & 0 & \nabla g^T & \nabla h^T \\
            0 & \Sigma & 0 & -I \\
            \nabla g & 0 & 0 & 0 \\
            \nabla h & -I & 0 & 0 \\
        \end{bmatrix} 
        \begin{bmatrix}
            \Delta x \\
            \Delta s \\
            -\Delta \lambda \\
            -\Delta \mu\\
        \end{bmatrix} = -
        \begin{bmatrix}
            \nabla f(x) + \nabla g(x)^T \lambda + \nabla h(x)^T \mu \\
            \mu - \rho I(s)^{-1} e \\
            g(x) \\
            h(x) + s \\
        \end{bmatrix}
    \end{equation}

    Interior point methods tend to converge very quickly to tight tolerances, exhibiting 
    quadratic convergence near the solution. However, they tend to be less amenable to 
    warm-starting than SQP or ALM. The most common implementation is the open-source 
    Ipopt solver \cite{wachter_implementation_2006}, which uses a filter approach together 
    with a feasibility restoration phase for improving globalization. For more details on
    interior point methods for solving LPs, QPs, and NLPs, see 
    \cite{nocedal_Numerical_2006}.

\end{document}