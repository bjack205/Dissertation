\documentclass[../root.tex]{subfiles}

\begin{document}
\chapter{Conclusion} \label{chap:conclusion}
\lettrine{R}{obots} have considerable potential to fulfill many meaningful roles 
in our society, across a broad range of applications and sectors of 
our economy. As robots move into the applications of today and tomorrow,
they need to be endowed with a greater ability to make meaningful 
decisions and operate at the limits of their desired objectives, 
whether that be gathering scientific information on another planet,
unloading pallets, delivering mail, or providing helpful care to 
the sick or the elderly. A key component in all of these applications 
is the ability to make and follow efficient motion plans.

The work detailed in this thesis has made significant contributions to
progressing the promising field of trajectory optimization. It has pushed the
limits of computational performance, allowing robots to plan faster or take
more into consideration in the plans they generate. It has provided
meaningful theoretical contributions that will aid the next generation of
roboticists develop better tools for systems that operate freely in
three-dimensional space. Additionally, it has suggested research directions
that better exploit modern parallel computers. While very powerful, the 
model-based algorithms proposed in this thesis are generally only as good 
as the underlying model itself. To address this limitation, this thesis
has also proposed a sample-efficient approach to learning an improved model by 
combining information from the real system with gradient information from an 
approximate prior model.


\end{document}