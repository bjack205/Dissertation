\documentclass[../root.tex]{subfiles}

\begin{document}
\chapter{Quaternion Error Maps} \label{app:quaternion_error_maps}

As detailed in Chapter \ref{chap:attitude}, when taking derivatives with
respect to quaternions, we need to have a mapping from the non-singular space
of unit quaternions to a local three-parameter representation, parameterizing
the plane tangent to the quaternion hypersphere at the group identity. As
mentioned in Sec. \ref{sec:Quaternion_Calculus}, there are several different
mappings that have been proposed in the literature. In this appendix, we 
provide more details about these mappings and show that, at the group identity,
all these mappings result in the same Jacobians, allowing the results in Sec. 
\ref{sec:Quaternion_Calculus} to apply regardless of the mapping used.

\section{Forward and Inverse Maps} 

Table \ref{tab:quat_error_maps} summarizes the four most common quaternion
error maps. Note that these definitions may be slightly different due to some
scaling parameters that have been added such that the derivatives in the
following sections are all consistent. The exponential map was scaled by a
factor of two while the MRP map was scaled by a factor of one half. Here we
define the forward map to be the map that takes the three-parameter
representation $\phi, g, p$, or $c$ and converts it back into a unit quaternion,
while the inverse map does the opposite. For notational convenience, we
define $s \in \R$ and $v \in \R^3$ to be the scalar and vector part of the
unit quaternion.

\begin{table}[h!]
    \centering
    \begin{tabular}{r l l}
        Map & Forward & Inverse \\
        \midrule
        Exponential 
            & $ \displaystyle q = \begin{bmatrix} \cos(\norm{\phi}) \\ \frac{\phi}{\norm{\phi}} \sin(\norm{\phi}) \end{bmatrix} $ 
            & $ \displaystyle \phi = \frac{v}{\norm{v}} \, \text{atan2}(\norm{v}, s)  $ \\
        Cayley
            & $ \displaystyle q = \frac{1}{1 + \norm{g}^2} \begin{bmatrix} 1 \\ g \end{bmatrix} $ 
            & $ \displaystyle g = \frac{v}{s} $ \\
        MRP
            & $ \displaystyle q = \frac{1}{1 + \norm{p}^2} \begin{bmatrix} 1 - \norm{p}^2 /4 \\ p \end{bmatrix} $
            & $ \displaystyle p = \frac{2}{1+s} \, v$ \\
        Vec
            & $ \displaystyle q = \begin{bmatrix} \sqrt{1 - \norm{c}^2} \\ c \end{bmatrix} $ 
            & $ \displaystyle c = v $
    \end{tabular}
    \caption{Basic quaternion error maps}
    \label{tab:quat_error_maps}
\end{table}

\section{Forward Jacobian}
As summarized in Sec. \ref{sec:quat_jacobian}, when taking the Jacobian of a function 
$y = h(q) : \Q \mapsto \R^p$, we apply a differential rotation $\phi$ to the input and
apply the chain rule:
\begin{equation} \label{eq:quat_gradient2}
    \nabla h(q) = \pdv{h}{q}  L(q) \pdv{\varphi}{\phi}
\end{equation}
Since $\phi$ is a differential rotation, we evaluate the Jacobian of the
forward map $\varphi$ as $\phi \rightarrow 0$. In Sec. \ref{eq:quat_jacobian}
we claimed that 
\begin{equation} \label{eq:quat_forward_identity}
    \lim_{\phi \rightarrow 0} \pdv{\varphi}{\phi} = H.
\end{equation}
We now prove that claim by deriving the Jacobians for the forward maps in
Table \ref{tab:quat_error_maps}.

\begin{table}[h!]
    \centering
    \begin{tabular}{r l}
        Map & Forward Jacobian \\
        \midrule
        Exponential 
            & $ \displaystyle \frac{1}{\norm{x}} \begin{bmatrix}
                -\norm{x} \sin(\norm{x}) \\
                (I - \frac{x x^T}{x^T x}) \sin(\norm{x}) + \frac{x x^T}{\norm{x}} \cos(\norm{x})
            \end{bmatrix}  $ \\
        Cayley 
            & $ \displaystyle (1+\norm{g}^2)^{\sfrac{\text{-}3}{2}} \begin{bmatrix} -g^T \\ (1+\norm{g}^2)I - g g^T \end{bmatrix} $ \\
        MRP 
            & $ \displaystyle (1 + \norm{p}^2)^{\text{-}2} \begin{bmatrix} -p^T \\ (1+\frac{1}{4} \norm{p}^2) I - \half p p^T \end{bmatrix} $ \\
        Vec 
            % & $ \displaystyle \begin{bmatrix} -(1 - \norm{c}^2)^{\sfrac{\text{-}1}{2}} \, c^T \\ I \end{bmatrix} $
            & $ \displaystyle \begin{bmatrix} \frac{-c^T}{\sqrt{1 - \norm{c}^2}} \\ I \end{bmatrix} $
    \end{tabular}
\end{table}
For the Cayley, MPR, and Vec maps, the identity \eqref{eq:quat_forward_identity} can 
be easily verified by plugging in $\zeros_3$ for $g$, $p$, and $c$, respectively.
The exponential map, however, requires a slightly more careful analysis. Using small
angle approximations, we can approximate the exponential map as 
\begin{equation} \label{eq:expm_approx}
    \begin{bmatrix}
        1 - \half \norm{v}^2 \\ \frac{\phi}{\norm{\phi}} \norm{\phi}
    \end{bmatrix} = \begin{bmatrix}
        1 - \half \norm{v}^2 \\ \phi
    \end{bmatrix}.
\end{equation}
The identity \eqref{eq:quat_forward_identity} can then be easily verified.

\section{Inverse Jacobian}
Similar to the forward Jacobian, when we take the Jacobian of a 
function $q' = f(q) : \Q \rightarrow \Q$ as in Sec. \ref{sec:quatquat_jacobian}, we 
apply a differential rotation to both the input and the output:
\begin{equation}
    f(q) \approx \pdv{}{\phi} \, \varphi^{-1}(L(q')^T f( L(q) \varphi(\phi)) \phi.
\end{equation}
We obtain the Jacobian by again applying the chain rule:
\begin{equation}
    \nabla f(q) = \pdv{\varphi^{-1}}{q} L(q')^T \pdv{f}{q} L(q) \pdv{\varphi}{\phi}.
\end{equation}
Note that as the differential value applied to the output goes to zero, the
term inside the inverse map $\varphi^{-1}$ goes to the quaternion identity 
$q_I = [1 \, 0 \, 0 \, 0]^T$ since $q' = f(q)$. Therefore, we only need to 
evaluate the Jacobian of the inverse map at the quaternion identity. We claimed
in Sec. \ref{sec:quatquat_jacobian} that 
\begin{equation} \label{eq:quat_inv_identity}
    \lim_{q \rightarrow q_I} \pdv{\varphi^{-1}}{q} = H^T.
\end{equation}
We now show that this is true for all of the maps in Table \ref{tab:quat_error_maps}
by deriving the Jacobians of the inverse maps, shown in Table \ref{tab:quat_invjac}.

\begin{table}[h!]
    \centering
    \begin{tabular}{r l}
        Map & Inverse Jacobian \\
        \midrule
        Exponential 
            & $ \displaystyle \frac{1}{(\norm{v}^2 + s^2)} \begin{bmatrix} 
                -v & 
                \frac{1}{\norm{v}}(\norm{v}^2 + s^2) (I - \frac{v v^T}{v^T v}) \text{atan2}(\norm{v}, s) + \frac{v v^T}{\norm{v}^2} s
            \end{bmatrix} $ \\
        Cayley 
            & $ \displaystyle \begin{bmatrix}
                -s^{-2} v & s^{-1} I
            \end{bmatrix} $ \\
        MRP 
            & $ \displaystyle \frac{1}{(1+s^2)^2} \begin{bmatrix}
                -v & (1+s) I
            \end{bmatrix} $ \\
        Vec
            & $ \displaystyle \begin{bmatrix} \zeros_3 & I_3 \end{bmatrix}  $
    \end{tabular}
    \caption{Inverse Quaternion Error Map}
    \label{tab:quat_invjac}
\end{table}
The identity \eqref{eq:quat_inv_identity} can again be easily verified for the Cayley,
MRP, and Vec maps by plugging in $s=1, v=\zeros_3$. As before, the exponential map requires 
a more careful analysis since plugging in $v=\zeros_3$ results in divisions by 0. Using the
first two terms of the Taylor expansion of $\text{arctan}(x) = x - \frac{x^3}{3}$, we 
get the following approximation for the inverse exponential map (also called the logarithmic
map):
\begin{equation}
    \begin{aligned}
        \varphi^{-1}_\text{exp}(\phi) &\approx \frac{v}{\norm{v}} 
            \left(\frac{\norm{v}}{s} - \frac{\norm{v}^3}{3s^3} \right) \\
        &= \frac{v}{s} \left(1 - \frac{\norm{v}^2}{3s^2} \right)
    \end{aligned}
\end{equation}
Taking the Jacobian of this approximation gives $\pdv*{\varphi^{-1}}{q} =
[\pdv*{}{s} \, \pdv*{v}]$, where
\begin{subequations}
    \begin{align}
        \pdv*{}{s} &= \frac{v}{s^2} \left( \frac{3 \norm{v}^2}{3 s^2} - 1 \right) \\
        \pdv*{}{v} &= \frac{1}{s}\left(1- \frac{\norm{v}^2}{3s^2} \right) I - \frac{2v v^T}{3s^3}
    \end{align}
\end{subequations}
which, when evaluate at $s=1, v=\zeros_3$, gives $H^T$, as expected.


\section{Second-Order Jacobian}
Here we provide a little more detail on deriving \eqref{eq:quat_hessian} in 
Sec. \ref{sec:quat_hessian}. Given a function $h(q) : \Q \mapsto \R$, we want 
an expression for the Hessian of $h$. Equivalently, we want to take the 
Jacobian of \eqref{eq:quat_gradient2}. Again applying the chain rule we get two
separate terms:
\begin{equation} \label{eq:quathess_chainrule}
    \nabla^2 h(q) = \pdv{\varphi}{\phi}^T L(q)^T \pdv[2]{h}{q} L(q) \pdv{\varphi}{\phi}
        + \pdv{}{\phi} \left( \pdv{\varphi}{\phi}^T L(q)^T \pdv{h}{q}^T \right)
\end{equation}
The second term is the second derivative of the forward map $\varphi$. To avoid 
dealing directly with the rank-3 tensor, we instead derive analytic expressions 
for Jacobians of the Jacobian-transpose-vector product:
\begin{equation}
    \nabla^2 \varphi(\phi, b) = \pdv{}{\phi} \left( \pdv{\varphi}{\phi}^T b \right).
\end{equation}
In Sec. \ref{sec:quat_hessian} we used the following identity, 
\begin{equation}
    \nabla^2 \varphi(\phi, b) = - w I_3
\end{equation}
where $w$ is the first element of the vector $b$. From
\eqref{eq:quathess_chainrule} we see that the vector $b$ is equal to $L(q)^T
\pdv{h}{q}^T$, where $\pdv{h}{q} \in \R^{1\times4}$ is the gradient with respect 
to $q$, treating $q$ as a standard vector in $\R^4$. Examining the definition 
of $L(q)$ in \eqref{eq:Lmult}, we see the first column is just $q$, so the 
first element of the vector $b$ is simply $\pdv{h}{q} q$. We then obtain the 
expression for the second term in \eqref{eq:quat_hessian}: $-(\pdv*{h}{q} q) I_3$.
Since these second-order expressions are more involved, we derive them in the 
subsections below. Again for notational clarity, we define $b = [w \; y^T]^T$, 
where $w \in \R, y \in \R^3$ correspond to the scalar and vector parts of the 
quaternion, $s$ and $v$.

\subsection{Exponential Map}
Since the full second-order expansion of the exponential map is complicated and 
needs to be approximated anyways for the same reasons as the previous sections,
we instead derive the second-order expansion for the small-angle approximation
\eqref{eq:expm_approx}.

\begin{equation}
    \begin{aligned}
        \nabla^2 \varphi_\text{exp}(\phi,b) &= \pdv{}{\phi} \left( -\phi w + y \right) \\
        &= -w I_3
    \end{aligned}
\end{equation}

\subsection{Cayley Map}
The second-order term for the Cayley map is:
\begin{equation} \label{eq:cayley_hess}
    \nabla^2 \varphi_\text{cay}(g,b) = 
        \left( 
            \left( g w + (g g^T y - (1+g^T g) y) \right) \frac{3 g^T}{1+g^T g} - 
            \left( I (w + g^T y) + g y^T - 2y g^T \right) 
        \right) (1 + g^T g)^{-3/2}
\end{equation}
Evaluating this expression at $g = \zeros_3$ we get $-w I_3$.

\subsection{MRP Map}
The second-order term for the MRP map is:
\begin{equation}
    \nabla^2 \varphi_\text{mrp}(p,b) = 
        \frac{1}{2(1 + \frac{1}{4}p^Tp)^3}
        \left(
            \left( 2p w + p p^T y - 2(1 + \frac{1}{4}p^Tp) \right) p^T - 
            (1 + \frac{1}{4}p^Tp) \left( 2w I_3 + p'y I_3 + p y^T - y p^T \right)
        \right).
\end{equation}
Which, when evaluated at $p = \zeros_3$, equals $-w I_3$.

\subsection{Vec Map}
The second-order term for the Vec map is:
\begin{equation}
    \nabla^2 \varphi_\text{vec}(p,b) = 
    -\frac{w}{(1-c^T c)^{\sfrac{3}{2}}} \left( c c^T + (1-c^T c) I \right).
\end{equation}
Which evaluated at $c = \zeros_3$ also equals $-w I_3$.


\end{document}